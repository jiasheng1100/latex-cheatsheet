\section{Implicature}
Tools to get to grips with frequent compositional structures in natural language (adj-n, adv-v, art-n, prep-np... combis), a higher-order logic
\subsection*{Grice's Maxims}
\emph{The cooperative principle}: contribution as required\\
\emph{Maxim of Quality}: nothing false or lacks evidence\\
\emph{Quantity}: as informative as required\\
Relation(or Relevance)\\
\emph{Manner}: clear and easy to understand\\
Failure to fulfill a maxim:\\
(i) quietly violate a maxim. Politician: Yes, this is what we stand for.\\
(ii) opt out from adhering to the maxim or the cooperative principle. Politician: I won’t answer this question.\\
(iii) a clash, impossible to adhere to one maxim without not adhering to another. Politician: We are still deciding on the matter. I’m hopeful that
yes, but I cannot tell you for sure.\\
(iv) flout a maxim. Politician: I personally think this is a good idea.
\subsection*{Conversational Implicature}
a type of pragmatic inference about what is said by the speaker (literal meaning) in relation to what they actually intend to convey (communicative intention).\\
\emph{Group A}: no maxim is violated. {\tiny A: C doesn’t seem to have a partner these days. B: He/she has been paying a lot of visits to New York lately. Implicature: He/she might have a partner in New York.}\\
\emph{Group B}: a maxim is violated, can be explained by a clash with another maxim. {\tiny A: Where does C live? B: Somewhere in the South of France. Implicature: I don’t know the exact name of the place where C lives.}\\
\emph{Group C}: exploitation, a maxim is flouted for the purpose of deliberately creating a conversational implicature. {\tiny Recommendation letter: Dear B, C’s command of English is excellent, and he has attended tutorials regularly. Kind regards, A. Implicature: I cannot recommend C as a philosopher.}\\
\subsection*{Types of Implicature}
Conversational Implicatures:\\
\emph{Particularized}: the intended inference depends on particular features of the specific context of the utterance. {\tiny A: C managed to brake his car and get arrested for arrousing
public annoyance when he was drunk last night.
B: Yeah, he is smart like that.}\\
\emph{Generalized: Scalar, Connectives, Indefinite}: does not depend
on specific features of the utterance
context, but is instead normally implied by
any use of the triggering expression in
ordinary contexts.\\
\emph{Scalar}: non-maximal degree modifiers. {\tiny The water is warm -> The water is not hot. John has most of the documents -> John does not have all of the documents}\\
\emph{Connectives}: sentence connectives. {\tiny Susan gave Peter the key and Peter opened the door. -> She gave him the key and then he opened the door. Peter is either Susan’s brother or her boyfriend -> The speaker does not know whether Peter is Susan’s brother or boyfriend.}\\
\emph{Indefinites}: indefinite article. {\tiny I walked into a house. -> The house was not my house.}\\
Conventional Implicatures: \\
not context-dependent or pragmatically explainable [in contrast to conversational implicatures], and must be learned on a word-by-word basis. (controversial, similar to presuppositions?){\tiny Alfred has still not come -> His arrival is expected. I was in Paris last spring too -> Some other person was in Paris last spring. Even Bart has passed the test -> Bart was among the least likely to pass the test}
\subsection*{Entailment}
1. whenever p is true, it is logically necessary that q is also true;\\
2. whenever q is false, it is logically necessary that p is also false;\\
3. these relations follow from the meanings of p and q, independent of the context of utterance\\
{\tiny I broke your Ming dynasty jar (lexical) -> Your Ming dynasty jar is broken. Hong Kong is warmer than Beijing (comparative) -> Beijing is cooler than HK}\\
\subsection*{Tests}
\begin{tabular}{c c c c}
 &Entailment&Convers. Implicature&\\
cancellable&no&yes&\\
suspendable&no&yes&\\
reinforceable&no&yes&\\
negation&no&no&\\
question&no&no&\\
\end{tabular}
\emph{Cancellation} {\tiny HK is warmer than BJ, but BJ is not cooler than HK (NO). There is a garage around the corner, but unfortunately you cannot buy petrol there (YES)}\\
\emph{Suspension} {\tiny HK is warmer than BJ, but I'm not sure if BJ is cooler than HK (NO). There is a garage around the corner, but I'm not sure if you cannot buy petrol there (YES)}\\
\emph{Reinforcement} {\tiny HK is warmer than BJ, and BJ is cooler than HK (NO). There is a garage around the corner, and you cannot buy petrol there (YES)}\\
\emph{Negation} {\tiny HK is not warmer than BJ (BJ is cooler than HK: NO). There is no garage around the corner (you cannot buy petrol there: NO)}\\
\emph{Question} {\tiny Is HK warmer than BJ? (BJ is cooler than HK: NO). Is there a garage around the corner? (you cannot buy petrol there: YES)}\\