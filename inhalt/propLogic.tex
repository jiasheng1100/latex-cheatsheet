\section{Propositional Logic}
Why formal logic? overcome ambiguity, determine relationships between meanings of sentences, determine meanings of setences, model compositionality, recursive system.

\subsection*{Definition}
Proposition: The meaning of a simple declarative sentence \\
Extensions: real-world situations they refer to. \\
Frege’s Generalization: The extension of a sentence S is its truth value \\
The proposition expressed by a sentence is the set of possible cases [situations] of which that sentence is true. \\
propositional variables: p, q, r...
propositional operators: $\neg, conjunction \land, disjunction \lor, XOR, \to, \leftrightarrow$

\subsection*{Syntax: Recursive Definition}
\begin{enumerate}
\item Propositional letters in the vocabulary of L are formulas in L.
\item If $\phi$ is a formula in L, then $\neg \phi$ is too.
\item If $\phi$ and $\psi$ are formulas in L, then $(\phi \land \psi), (\phi \lor \psi), (\phi \to \psi), (\phi \leftrightarrow \psi)$ are too.
\item Only that which can be generated by the clauses (i)-(iii) in a finite
number of steps is a formula in L.
\end{enumerate}
invalid: $\neg (\neg \neg p), \neg ((p \land q))$ \\

\subsection*{Construction Trees}
\begin{forest}
[$(\neg (p \lor q) \to \neg \neg q) \leftrightarrow r$ $(iii. \leftrightarrow)$
	[$(\neg (p \lor q) \to \neg \neg q)$ $(iii. \to)$
		[$\neg (p \lor q)$ $(ii)$
			[$p \lor q$ $(iii. \lor)$
				[$p$ (i)]
				[$q$ (i)]
			]			
		]
		[$\neg \neg q$ (ii)
			[$\neg q$ (ii)
				[$q$ (i)]
			]
		]
	]
	[$r(i)$]
]
\end{forest}

\subsection*{Valuation Functions}
For every valuation V and for all formulas $\phi$: \\
$V(\phi \leftrightarrow \psi) = 1 iff V(\phi) = V(\psi)$.
