\section{Propositional Logic}
Why formal logic? {\tiny overcome ambiguity, determine relationships between meanings of sentences, determine meanings of setences, model compositionality, recursive system.}
\subsection*{Definition}
Proposition {\tiny The meaning of a simple declarative sentence. The proposition expressed by a sentence is the set of possible cases [situations] of which that sentence is true.} \\
Extensions {\tiny real-world situations they refer to} \\
Frege’s Generalization {\tiny The extension of a sentence S is its truth value} \\
{\scriptsize propositional variables: p, q, r\\
propositional operators: $\neg, \land, \lor, XOR, \to, \leftrightarrow$}
\subsection*{Syntax: Recursive Definition}
{\scriptsize (i) Propositional letters in the vocabulary of L are formulas in L.\\
(ii) If $\phi$ is a formula in L, then $\neg \phi$ is too.\\
(iii) If $\phi$ and $\psi$ are formulas in L, then $(\phi \land \psi), (\phi \lor \psi), (\phi \to \psi), (\phi \leftrightarrow \psi)$ are too.\\
(iv) Only that which can be generated by the clauses (i)-(iii) in a finite number of steps is a formula in L.\\}
invalid: $\neg (\neg \neg p), \neg ((p \land q))$ 
\subsection*{Construction Trees}
\begin{forest}
[$(\neg (p \lor q) \to \neg \neg q) \leftrightarrow r$ $(iii. \leftrightarrow)$
	[$(\neg (p \lor q) \to \neg \neg q)$ $(iii. \to)$
		[$\neg (p \lor q)$ $(ii)$
			[$p \lor q$ $(iii. \lor)$
				[$p$ (i)]
				[$q$ (i)]
			]			
		]
		[$\neg \neg q$ (ii)
			[$\neg q$ (ii)
				[$q$ (i)]
			]
		]
	]
	[$r(i)$]
]
\end{forest}
\subsection*{Valuation Functions}
{\tiny For every valuation V and for all formulas} $\phi$: $V(\phi \leftrightarrow \psi) = 1 iff V(\phi) = V(\psi)$.