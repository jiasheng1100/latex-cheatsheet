\section{Presupposition}
information which is linguistically encoded as being part of the common ground at the time of utterance. {\tiny common ground: everything that both the speaker and hearer know or believe, and know that they have in common.}\\
Statement A and presupposition B: (i) if A is true, then B is true (ii) if A is false, then B is still true.\\
\subsection*{Presupposition Triggers}
\emph{Definite descriptions}: definite noun phrases, possessive phrases, restrictive relative clauses. e.g. the, my, the man who can fly\\
\emph{Factive predicates}: regret, be aware, realize, be sorry, know\\
\emph{Implicative Predicates}: manage to, forget to (presupposes other predicates, e.g. try to, intend to, to be true)\\
\emph{Aspectual Predicates}: express the beginning, stopping, continuing of events. e.g. stopped, has begun, continues to, resume\\
\emph{Temporal clauses}: before, after, by the time, while\\
\emph{Counterfactuals}: If I were..., If you had not...I would not have...\\
\emph{Comparisons}: as unreliable as, as old as...\\
\emph{Scalars}: more, some...\\
\subsection*{Accomodation and Failure}
\emph{Accomodation}: hearers accept the presupposition as true, or they might ask for confirmation to “officially” establish the presupposition as common ground\\
\emph{Failure}: the hearer rejects the presupposition\\
\emph{Group C}: exploitation, a maxim is flouted for the purpose of deliberately creating a conversational implicature. {\tiny Recommendation letter: Dear B, C’s command of English is excellent, and he has attended tutorials regularly. Kind regards, A. Implicature: I cannot recommend C as a philosopher.}\\