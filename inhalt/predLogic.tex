\section{Predicate Logic}
{\tiny Introduce constants and variables representing invididuals and predicates to capture the main structural building blocks of sentences. Introduce quantifiers to allow for quantified statements.}
\subsection*{Definition}
{\tiny constant symbols: a, b, c\\
variable symbols: x, y, z\\
n-ary/n-place predicate symbols: A, B, C, reflect relations between n elements (n>0)\\
connectives:$\neg, \land, \lor, \to, ...$ \\
quantifiers: $\forall, \exists$ \\
round brackets (), equal sign =}
\subsection*{Syntax: Recursive Definition}
{\scriptsize (i) If A is an n-ary predicate letter in the vocabulary of L, and each of t1,..., tn is a constant or a variable in the vocabulary of L, then At1,..., tn is a formula in L.\\
(ii) If $\phi$ is a formula in L, then $\neg \phi$ is too.\\
(iii) If $\phi$ and $\psi$ are formulas in L, then $(\phi \land \psi), (\phi \lor \psi), (\phi \to \psi), (\phi \leftrightarrow \psi)$ are too.\\
(iv) If $\phi$ is a formula in L and x is a variable, then $\forall x \phi$ and $\exists x \phi$ is too.
(v) Only that which can be generated by the clauses (i)-(iv) in a finite
number of steps is a formula in L.}\\
invalid: $a, A, \forall(Axy)$ 
\subsection*{Construction Trees}
\begin{forest}
[$\forall_x \forall_y ((A_{xy} \land B_y) \to \exists_x A_{xb})$ $(iv. \forall)$
	[$\forall_y ((A_{xy} \land B_y) \to \exists_x A_{xb})$ $(iv. \forall)$
		[$(A_{xy} \land B_y) \to \exists_x A_{xb}$ $(iii. \to)$
			[$A_{xy} \land B_y$ $(iii. \land)$
				[$A_{xy}$ (i)]
				[$B_y$ (i)]
			]
			[$\exists_x A_{xb}$ $(iv. \exists)$
				[$A_{xb}$ (i)]
			]
		]
	]
]
\end{forest}
\subsection*{Semantics}
domain (D): set of entities \\
interpretation functions I = $\{<m,e>, <s,e>, <v,e> \}$, I(m) = e, I(s) = e, I(v) = e. \\
model M: consists of a domain D and an interpretation function I which conforms to:
{\tiny (i) if c is a constant in L, then I(c) $\in$ D. (ii) if B is an n-ary prpedicate letter in L, then I(B) $\subset$ D\\}
valuation function $V_M$: \\
{\tiny If Aa1,...,an is an atomic sentence in L, then $V_M$(Aa1,...,an) = 1 iff <I(a1),...,I(an)> $\in$ I(A). \\
... \\
$V_M$($\forall x \phi$) = 1 iff $V_M$([c/x]$\phi$) = 1 for all constants c in L. \\
$V_M$($\exists x \phi$) = 1 iff $V_M$([c/x]$\phi$) = 1 for at least one constant c in L. \\
If $V_M$($\phi$) = 1, then $\phi$ is said to be true in model M.}
\subsection*{Formula vs. Sentence}
A sentence is a formula in L which lacks free variables. \\
Sentence: Aa, $\forall x (Fx)$, $\forall x (Ax \to \exists y By)$ \\
Not a sentence (but Formula): Ax, Fx, $Ax \to \exists y By$