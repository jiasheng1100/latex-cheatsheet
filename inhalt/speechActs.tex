\section{Speech Acts}
information which is linguistically encoded as being part of the common ground at the time of utterance. {\tiny common ground: everything that both the speaker and hearer know or believe, and know that they have in common.}\\
Statement A and presupposition B: (i) if A is true, then B is true (ii) if A is false, then B is still true.
\subsection*{Performatives}
indicative mood and present tense, use of performative verb(e.g. sentence, declare, confer, invite, request, order, accuese...), active void of a first person subject, usage of performative adverb ´hereby’.\\
\emph{Felicity Conditions}:\\
A.1 Conventionality Condition\\
{\tiny accepted conventional procedure}\\
A.2 Appropriateness Cond.\\
{\tiny appropriate persons and circumstances}\\
B.1 Correctness Cond.\\
B.2 Completeness Cond.\\
{\tiny procedure executed correctly and completely}\\
C.1 Sincerity Cond.\\
{\tiny person must intend so}\\
C.2 Subsequent Conduct Cond.\\
{\tiny person must subsequently conduct so}\\
\emph{Violations of Conditions}:\\
Misfire: conditions under A-B violated\\
Abuse: conditions under C violated\\
All sentences can be paraphrased as performatives
\subsection*{Speech Acts}
\emph{Locutionary Art}: The act of performing an utterance (phonetically and grammatically) {\tiny production and pronounciation of the sentence, given knowledge of the vocabulary and grammar, and the referent}\\
{\tiny (i)phonetic act: uttering certain speech sounds with the speech aparatus. (ii)phatic act: use of certain strings of speech sounds belonging to a certain vocabulary and conforming to a certain grammar. (iii)rhetic act: uttering the respective words with a certain "more or less" definite sense and reference}\\
\emph{Illocutionary Act}: The act of performing a statement, question, command, etc. by means of its conventional force (i.e. what is the locutionary act used for?){\tiny ask or answer questions, assure or warn, announce a verdict or an intention, protest against, command, give advice...}\\
\emph{Perlocutionary Act}: The act of effecting the audience in a particular way {\tiny stop/annoy/persuade... someone}\\
\subsection*{Direct and Indirect Speech Acts}
\emph{Direct}: the type of sentence (grammatical form) matches the type of illocutionary force\\ {\tiny Declarative -> Statement (It is raining); Interrogative -> Question (Is it raining?); Imperative -> Command (Make it rain!)}\\
\emph{Indirect}: an utterance whose form does not reflect the intended illocutionary force {\tiny I want you to leave now (Declarative -> command); I would like to have a cup of tea (Declarative -> request); Can you pass me the salt? (Interrogative -> command); Isn't this a beautiful day? (Interrogative -> statement)}\\
{\tiny how does the addressee figure out the intended illocutionary force -> the Gricean method of calculating implicatures}