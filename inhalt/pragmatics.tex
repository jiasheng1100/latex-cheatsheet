\section{Introduction to Pragmatics}
Semantics: word meaning, sentence meaning
Pragmatics: utterance meaning
\subsection*{Definitions}
\emph{Anomaly Definition}:study of those principles that will account for why a certain set of sentences are anomalous, or not possible
utterances.\\
\emph{Functional}: attempts to explain facets of linguistic structure by reference to non-linguistic pressures and causes.\\
\emph{Context}: part of performance, explicate the reasoning of speakers and hearers in working out the correlation in a context of a sentence token with a proposition.\\
\emph{Grammaticalization}: study of those relations between language and context that are grammaticalized, or encoded in the structure of a language.\\
\emph{Truth-Conditional}: those aspects of the meaning of utterances which cannot be accounted for by straightforward reference to the truth conditions of the sentences uttered.\\
\emph{Inter-Relation}: interation of context-dependent aspects of language structure and principles of language usage, relations between
language and context\\
\emph{Appropriateness/Felicity}: study of the ability of language users to
pair sentences with the contexts in which they would be
appropriate.\\
\emph{List}: study of deixis (at least in part), implicature, presupposition, speech acts, and aspects of
discourse structure.\\
More promising: Inter-Relation, Truth-Conditional