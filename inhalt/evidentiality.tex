\section{Evidentiality}
covers the way in which information was acquired, without necessarily relating to the degree of speaker’s certainty concerning the statement or whether it is true or not. To be considered as an evidential, a morpheme has to have ‘source of information’ as its core meaning; that is, the unmarked, or default
interpretation.
\subsection*{Definition}
\emph{1st claim}: It is a “linguistic category”, i.e. a grammatical category with grammatical markers (same as for modality).\\
\emph{2nd claim}: These evidential markers have source of information as their core meaning.\\
markers can develop polysemy, e.g. tense marking and evidential marking, can be used recursively without being redundant\\
\emph{3rd claim}: Evidentiality is not “necessarily relating to the degree of speaker’s certainty”, i.e. it is distinct from epistemic modality.
\subsection*{Evidentiality vs. Epistemic Modality}
There is good evidence that evidential markers in a number of languages do not contribute to propositional content but function as illocutionary modifiers, and so must be distinct from epistemic modality.\\
\emph{Negation Test}: If negation can scope over the evidential marker, then the evidential marker is considered to contribute to the truth-conditional content.\\
\emph{Challenge Test}: The hearer can challenge the truth of the statement of the
speaker given more direct evidence, but the source of information cannot be challenged.\\
Two types of evidentials:\\
\emph{Illocutionary}: markers of evidentiality that do not contribute to the truth-conditional content, but that “add to or modify the sincerity conditions of the [speech] act”\\
\emph{Propositional}: markers of evidentiality that also contribute to the truth-conditional content. e.g. Es soll regnen.
\subsection{Cross-Linguistic Variation}
Semantic distinctions of evidentiality: no grammatical evidentials/indirect only/direct and indirect\\
Coding of Evidentiality: no grammatical evidentials/verbal affix or clitic/part of the tense system/separate particle/modal morpheme/mixed