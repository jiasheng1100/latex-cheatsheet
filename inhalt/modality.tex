\section{Modality}
A category of linguistic meaning having to do with the expression of possibility and necessity. 
\subsection*{Modal Strength(Force)}
Statements can express stronger or weaker commitment to the truth of base proposition. \\
\emph{High}: Arthur must/has to be home.\\
\emph{Medium}: Arthur should be home.\\
\emph{Low}: Arthur might/could be home.
\subsection*{Modal Type(Flavor)}
\emph{Epistemic} modality: relative to speaker's knowledge of the situation\\
John didn’t show up for work. He must be sick. (spoken by co-worker)\\
The older students might/may(?)
leave school early (unless the
teachers watch them carefully).\\
It has to be raining. [Seeing people outside with umbrellas]\\
\emph{Deontic}: relative to authoritative person or code of conduct\\
John didn’t show up for work. He must be sick. (spoken by boss) \\
Visitors have to leave by 6pm.\\
\emph{Dynamic}: Concerned with properties and dispositions of persons\\
John has to sneeze.\\
Anne is very strong. She can list this table. \\
\emph{Teleological}: achieving goals or serving a purpose\\
To get home in time, you have to take a taxi.\\
Anne must be in Paris at 5pm. She can/must take the train to go there.
\subsection*{Polysemy Controversy}
In some languages, modal auxiliaries can be used for different types of modality.\\
Ambiguity(polysemy) vs. Indeterminacy\\
\emph{Contradiction test} (If a sentence of the form \emph{X but not X} can be true, then expression must be ambiguous.\\
e.g. They are not children any more, but they are still my
children. \\
John must be sick, but he must not be sick.\\
John can be sick, but he cannot be sick...\\
If considered non-contradictory, then the modal auxiliaries are polysemous with regards to modal type\\
\emph{Adverbial Phrase Test}: \\
e.g. Dynamic: (In view of his physical abilities,) John can lift 200 kg.\\
If adverbial phrases in parentheses are not redundant, type of modality is not lexically specidied but inferred from context, i.e. indeterminate
\subsection*{Modal Logical Operators}
$\diamondsuit p$: it is possible that p...\\
$\square p$: it is necessary that p...\\
modality as quantification over possible worlds: $\diamondsuit p \equiv \exists w[w \in p]$, $\square p \equiv \forall w[w \in p]$\\
Modal propositional logic: add one more syntactic clause to the syntax of propositional logic: \\
(v) if $\phi$ is a formula in L, then $\square \phi$ and $\diamondsuit \phi$ are too.\\
valid formulas: $\square \diamondsuit p$, $\neg \diamondsuit (p \land q)$, $p \to \square \diamondsuit p$\\
Fundamental tautologies:\\
$\diamondsuit \phi \leftrightarrow \neg \square \neg \phi$: something is possible if and only if it is
not the case that it is necessarily not the case\\
$\square \phi \leftrightarrow \neg \diamondsuit \neg \phi$: something is necessary if and only if it is not the case that it is possibly not the case.\\
\subsection*{Modality and Truth-Conditions}
Both epistemic and root modality can be part of the proposition and contribute to its truth conditions.\\
\emph{Challenge Test}: Is the epistemic modal marker part of what can be challenged about a proposition?\\
A: John profited from the old man’s death, he must
be the murderer.
B: That’s not true; he could be the murderer, but he
doesn’t have to be.\\
\emph{Yes-No Question Test}: Can the epistemic modal marker be the focus of a yes-no question?\\
A: Must John be the murderer?
B: Yes, he must. or: No, he doesn’t have to be.\\
\emph{Negation Test}: does negation scope over and hence include
the modal marker as part of the negated proposition?\\
Smith cannot be the candidate. $\neg \diamondsuit p \surd$\\
Smith might not be the candidate. $\diamondsuit \neg p$
\subsection{Cross-Linguistic Variation}
Epistemic possibility: verbal constructions/affixes on verbs/other\\
Situational possibility: affixes on verbs/verbal constructions/other